\documentclass[11pt]{article}
\usepackage{amsfonts,amssymb,amsmath,amsthm}
\usepackage{mathrsfs}
\usepackage{fullpage}

\newtheorem{theorem}{Theorem}%[section]
\newtheorem{lemma}[theorem]{Lemma}
\newtheorem{proposition}[theorem]{Proposition}
\newtheorem{remark}[theorem]{Remark}
\newtheorem{procedure}[theorem]{Procedure}
\newtheorem{definition}[theorem]{Definition}
\newtheorem{conjecture}[theorem]{Conjecture}
\newtheorem{observation}[theorem]{Observation}
\newtheorem{construction}[theorem]{Construction}
\newtheorem{claim}[theorem]{Claim}
\newtheorem{fact}[theorem]{Fact}
\newtheorem{notation}[theorem]{Notation}

\newtheorem{statement}{Statement}%[section]
\newtheorem{problem}[statement]{Problem}
\newtheorem{question}[statement]{Question}

\newtheorem{case}{Case}
\newtheorem{example}{Example}
\newtheorem{note}{Note}

\newcommand{\pow}[1]{\mathscr{P}(#1)}

\renewcommand{\labelenumi}{\emph{(\roman{enumi})}}
\setcounter{statement}{53}


\begin{document}

\author{Robert Ritchie}
\title{MTH 331 -- Statement 54}
\date{\today}

\maketitle

\begin{statement}
Let $x$ be a non-zero real number. If $x+1/x$ is an integer, then for all integers $n$ with $n\geq0$, $x^n+1/x^n$ is an integer. 
\end{statement}

\begin{proof}
\begin{enumerate}
\item $n=0$, $x^0+1/x^0=2$ is an integer
\item $n=1$, $x^1+1/x^1=x+1/x$
\item Let $x+1/x$ be an integer. For all $n\geq2$, if, for all $0\leq k\leq n-1$, $x^k+1/x^k$ is an integer then $x^n+1/x^n$ is an integer. 
Suppose for all k, $x^k+1/x^k$ is an integer.
\begin{align*}
(x+1/x)(x^{k-1}+1/x^{k-1})&=x^k+x^{2-k}+x^{k-2}+x^{-k}\\
&=x^k+x^{-k}+x^{k-2}+x^{-(k-2)}\\
&=x^k+1/x^k+x^{k-2}+1/x^{k-2}\\
\rightarrow x^{k-2}+1/x^{k-2}&=(x+1/x)(x^{k-1}+1/x^{k-1})-(x^k+1/x^k)
\end{align*}
We know that $x+1/x$,$x^{k}+1/x^{k}$  and $x^{k-1}+1/x^{k-1}$ are integers and they imply that $x^{k-2}+1/x^{k-2}$ is an integer thus $x^n+1/x^n$ is an integer.
\end{enumerate}
\end{proof}

\end{document} 