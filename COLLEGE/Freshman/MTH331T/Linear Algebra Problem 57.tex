\documentclass[11pt]{article}
\usepackage{amsfonts,amssymb,amsmath,amsthm}
\usepackage{mathrsfs}
\usepackage{fullpage}

\newtheorem{theorem}{Theorem}%[section]
\newtheorem{lemma}[theorem]{Lemma}
\newtheorem{proposition}[theorem]{Proposition}
\newtheorem{remark}[theorem]{Remark}
\newtheorem{procedure}[theorem]{Procedure}
\newtheorem{definition}[theorem]{Definition}
\newtheorem{conjecture}[theorem]{Conjecture}
\newtheorem{observation}[theorem]{Observation}
\newtheorem{construction}[theorem]{Construction}
\newtheorem{claim}[theorem]{Claim}
\newtheorem{fact}[theorem]{Fact}
\newtheorem{notation}[theorem]{Notation}

\newtheorem{statement}{Statement}%[section]
\newtheorem{problem}[statement]{Problem}
\newtheorem{question}[statement]{Question}

\newtheorem{case}{Case}
\newtheorem{example}{Example}
\newtheorem{note}{Note}

\newcommand{\pow}[1]{\mathscr{P}(#1)}

\renewcommand{\labelenumi}{\emph{(\roman{enumi})}}
\setcounter{statement}{-1}


\begin{document}

\author{Robert Ritchie}
\title{MTH 331 -- Problem 57-1}
\date{\today}

\maketitle

If S is the stretching transformation on $\mathbb{R}^{2}$, defined by \begin{center}
$S(x,y)=(7x,7y)$, 
\end{center} and T is the transformation on $\mathbb{R}^2$ defined by \begin{center} $T(x,y)=(2x+3y,7x-5y)$,\end{center} do S and T commute?
\begin{proof} 
Let $(x,y)\in\mathbb{R}^2$
\begin{align*}
TS(x,y)&=T(7x,7y)\\
&=(14x+21y,49x-35y)\\
&=S(2x+3y,7x-5y)\\
&=ST(x,y)
\end{align*}

\end{proof}

\end{document} 