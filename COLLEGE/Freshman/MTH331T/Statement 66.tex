\documentclass[11pt]{article}
\usepackage{amsfonts,amssymb,amsmath,amsthm}
\usepackage{mathrsfs}
\usepackage{fullpage}

\newtheorem{theorem}{Theorem}%[section]
\newtheorem{lemma}[theorem]{Lemma}
\newtheorem{proposition}[theorem]{Proposition}
\newtheorem{remark}[theorem]{Remark}
\newtheorem{procedure}[theorem]{Procedure}
\newtheorem{definition}[theorem]{Definition}
\newtheorem{conjecture}[theorem]{Conjecture}
\newtheorem{observation}[theorem]{Observation}
\newtheorem{construction}[theorem]{Construction}
\newtheorem{claim}[theorem]{Claim}
\newtheorem{fact}[theorem]{Fact}
\newtheorem{notation}[theorem]{Notation}

\newtheorem{statement}{Statement}%[section]
\newtheorem{problem}[statement]{Problem}
\newtheorem{question}[statement]{Question}

\newtheorem{case}{Case}
\newtheorem{example}{Example}
\newtheorem{note}{Note}

\newcommand{\pow}[1]{\mathscr{P}(#1)}

\renewcommand{\labelenumi}{\emph{(\roman{enumi})}}
\setcounter{statement}{66}


\begin{document}

\author{Robert Ritchie}
\title{MTH 331 -- Statement 67}
\date{\today}

\maketitle

\begin{statement}
$\{n\in\mathbb{Z}:15\mid n\} \cap \{ n\in\mathbb{Z}:2\mid n \}  \subseteq \{ n\in\mathbb{Z}:10\mid n \}$
\end{statement}

\begin{proof}
Let $n\in\mathbb{Z}$.
Suppose $15\mid n$ and $2\mid n$.
$\exists x \in \mathbb{Z}$ such that $n=15x$
\begin{align*}
n&=15x\\
\Leftrightarrow n&=5(3x)\\
\Rightarrow 5 &
\mid n
\end{align*}
$5 \mid n \wedge 2 \mid n \Rightarrow 10 \mid n$ (by statement 28)
\end{proof}

\end{document} 