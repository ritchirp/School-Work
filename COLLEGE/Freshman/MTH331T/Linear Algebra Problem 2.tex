\documentclass[11pt]{article}
\usepackage{amsfonts,amssymb,amsmath,amsthm}
\usepackage{mathrsfs}
\usepackage{fullpage}

\newtheorem{theorem}{Theorem}%[section]
\newtheorem{lemma}[theorem]{Lemma}
\newtheorem{proposition}[theorem]{Proposition}
\newtheorem{remark}[theorem]{Remark}
\newtheorem{procedure}[theorem]{Procedure}
\newtheorem{definition}[theorem]{Definition}
\newtheorem{conjecture}[theorem]{Conjecture}
\newtheorem{observation}[theorem]{Observation}
\newtheorem{construction}[theorem]{Construction}
\newtheorem{claim}[theorem]{Claim}
\newtheorem{fact}[theorem]{Fact}
\newtheorem{notation}[theorem]{Notation}

\newtheorem{statement}{Statement}%[section]
\newtheorem{problem}[statement]{Problem}
\newtheorem{question}[statement]{Question}

\newtheorem{case}{Case}
\newtheorem{example}{Example}
\newtheorem{note}{Note}

\newcommand{\pow}[1]{\mathscr{P}(#1)}

\renewcommand{\labelenumi}{\emph{(\roman{enumi})}}



\begin{document}

\author{Robert Ritchie}
\title{MTH 331 -- Problem 2}
\date{\today}

\maketitle
Problem 2: If a new addition for real numbers, denoted by the temporary symbol $\boxplus$, is defined by $a \boxplus b = 2a + b$, is $\boxplus$ associative? \\
$(a \boxplus b)\boxplus c = 4a + 2b + c$\\
 $a \boxplus (b \boxplus c) = 2a +2b+ c$\\
  $4a+2b+c \neq 2a+2b+c$ so $\boxplus$ is not associative.
\end{document}