\documentclass[11pt]{article}
\usepackage{amsfonts,amssymb,amsmath,amsthm}
\usepackage{mathrsfs}
\usepackage{fullpage}

\newtheorem{theorem}{Theorem}%[section]
\newtheorem{lemma}[theorem]{Lemma}
\newtheorem{proposition}[theorem]{Proposition}
\newtheorem{remark}[theorem]{Remark}
\newtheorem{procedure}[theorem]{Procedure}
\newtheorem{definition}[theorem]{Definition}
\newtheorem{conjecture}[theorem]{Conjecture}
\newtheorem{observation}[theorem]{Observation}
\newtheorem{construction}[theorem]{Construction}
\newtheorem{claim}[theorem]{Claim}
\newtheorem{fact}[theorem]{Fact}
\newtheorem{notation}[theorem]{Notation}

\newtheorem{statement}{Statement}%[section]
\newtheorem{problem}[statement]{Problem}
\newtheorem{question}[statement]{Question}

\newtheorem{case}{Case}
\newtheorem{example}{Example}
\newtheorem{note}{Note}

\newcommand{\pow}[1]{\mathscr{P}(#1)}

\renewcommand{\labelenumi}{\emph{(\roman{enumi})}}



\begin{document}

\author{Robert Ritchie}
\title{MTH 331 -- Problem 99}
\date{\today}

\maketitle
Let $f: \mathbb{Z} \rightarrow \mathbb{Z} \times \mathbb{Z}$ given by $f(n)=(2n+1,n-4)$ Is $f$ injective? Is $f$ surjective?
\begin{statement}
For f to be injective it must satifiy the following statement: $\forall a_{1},a_{2} \in \mathbb{Z}$ if $f(a_{1})=f(a_{2})$ then $a_{1}=a_{2}$
\end{statement}

\begin{proof}
Let $a_{1},a_{2} \in \mathbb{b}$. Suppose $f(a_{1})=f(a_{2})$.
\begin{align*}
&   f(a_{1})=f(a_{2}) \\
\Leftrightarrow &(2a_{1}+1,a_{1}-4)= (2a_{2}+1,a_{2}-4) \\
\Leftrightarrow &2a_{1}+1=2a_{2}+1 \wedge a_{1}-4=a_{2}-4\\ 
\Leftrightarrow &a_{1}=a_{2} \wedge a_{1}=a_{2}\end{align*}
\end{proof}
\begin{statement}
If f is not surjective, it must satisfy the following statement: $\exists (b,c) \in \mathbb{Z} \times \mathbb{Z} \quad \forall a \in \mathbb{Z}$ such that $f(a)\neq (b,c)$.
\end{statement}
\begin{proof}
\begin{align*}
Let \quad &b=-1 \quad c=-1\\
&b=-1 \rightarrow a=-1,\quad c=-1 \rightarrow a=3\\
 &so\: (-1,-1) \not\in f
\end{align*}
\end{proof}
\end{document} 